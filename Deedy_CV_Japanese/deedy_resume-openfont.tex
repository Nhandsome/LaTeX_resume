%%%%%%%%%%%%%%%%%%%%%%%%%%%%%%%%%%%%%%%
% Deedy - One Page Two Column Resume
% LaTeX Template
% Version 1.2 (16/9/2014)
%
% Original author:
% Debarghya Das (http://debarghyadas.com)
%
% Original repository:
% https://github.com/deedydas/Deedy-Resume
%
% IMPORTANT: THIS TEMPLATE NEEDS TO BE COMPILED WITH XeLaTeX
%
% This template uses several fonts not included with Windows/Linux by
% default. If you get compilation errors saying a font is missing, find the line
% on which the font is used and either change it to a font included with your
% operating system or comment the line out to use the default font.
% 
%%%%%%%%%%%%%%%%%%%%%%%%%%%%%%%%%%%%%%
% 
% TODO:
% 1. Integrate biber/bibtex for article citation under publications.
% 2. Figure out a smoother way for the document to flow onto the next page.
% 3. Add styling information for a "Projects/Hacks" section.
% 4. Add location/address information
% 5. Merge OpenFont and MacFonts as a single sty with options.
% 
%%%%%%%%%%%%%%%%%%%%%%%%%%%%%%%%%%%%%%
%
% CHANGELOG:
% v1.1:
% 1. Fixed several compilation bugs with \renewcommand
% 2. Got Open-source fonts (Windows/Linux support)
% 3. Added Last Updated
% 4. Move Title styling into .sty
% 5. Commented .sty file.
%
%%%%%%%%%%%%%%%%%%%%%%%%%%%%%%%%%%%%%%%
%
% Known Issues:
% 1. Overflows onto second page if any column's contents are more than the
% vertical limit
% 2. Hacky space on the first bullet point on the second column.
%
%%%%%%%%%%%%%%%%%%%%%%%%%%%%%%%%%%%%%%


\documentclass[]{deedy-resume-openfont}
\usepackage{fancyhdr}
 
\pagestyle{fancy}
\fancyhf{}
 
\begin{document}

%%%%%%%%%%%%%%%%%%%%%%%%%%%%%%%%%%%%%%
%
%     LAST UPDATED DATE
%
%%%%%%%%%%%%%%%%%%%%%%%%%%%%%%%%%%%%%%
% \lastupdated

%%%%%%%%%%%%%%%%%%%%%%%%%%%%%%%%%%%%%%
%
%     TITLE NAME
%
%%%%%%%%%%%%%%%%%%%%%%%%%%%%%%%%%%%%%%

\namesection{韓 }{範錫}{
ハン      ボム ソク\\[5pt]
\urlstyle{same}\href{mailto:han.beom.seok.707@s.kyushu-u.ac.jp}{han.beom.seok.707@s.kyushu-u.ac.jp}\\
\href{https://nhandsome.github.io/}{nhandsome.github.io} |  \href{https://www.linkedin.com/in/beomseok-han/}{linkedin.com/in/beomseok-han}
}

%%%%%%%%%%%%%%%%%%%%%%%%%%%%%%%%%%%%%%
%
%     COLUMN ONE
%
%%%%%%%%%%%%%%%%%%%%%%%%%%%%%%%%%%%%%%
\sectionsep
\begin{minipage}[t]{0.44\textwidth} 

%%%%%%%%%%%%%%%%%%%%%%%%%%%%%%%%%%%%%%
%     EDUCATION
%%%%%%%%%%%%%%%%%%%%%%%%%%%%%%%%%%%%%%

\section{学歴} 

\subsection{九州大学 | 日本}
\descript{大学院システム情報科学府・修士}
\location{2020年4月 - 2022年3月(見込み)}
情報学専攻・自然言語処理研究室\\
データサイエンスコース
\sectionsep

\subsection{弘益大学 | 韓国}
\descript{情報・コンピューターエンジニアリング・学士}
\location{2008年3月 - 2015年2月}
インダストリアルエンジニアリング専攻\\
\location{ Cum. GPA: 4.01 / 4.5}
\sectionsep

%%%%%%%%%%%%%%%%%%%%%%%%%%%%%%%%%%%%%%
%     LINKS
%%%%%%%%%%%%%%%%%%%%%%%%%%%%%%%%%%%%%%

% \section{Links} 
% Github:// \href{https://github.com/deedydas}{\bf deedydas} \\
% LinkedIn://  \href{https://www.linkedin.com/in/debarghyadas}{\bf debarghyadas} \\
% YouTube://  \href{https://www.youtube.com/user/DeedyDash007}{\bf DeedyDash007} \\
% Twitter://  \href{https://twitter.com/debarghya_das}{\bf @debarghya\_das} \\
% Quora://  \href{https://www.quora.com/Debarghya-Das}{\bf Debarghya-Das}

%%%%%%%%%%%%%%%%%%%%%%%%%%%%%%%%%%%%%%
%     COURSEWORK
%%%%%%%%%%%%%%%%%%%%%%%%%%%%%%%%%%%%%%

\section{スキル}
\subsection{プログラミング}
Python3 \textbullet{} Jupyter Notebook \textbullet{} Selenium \textbullet{} \\Java \textbullet{} Javascript \textbullet{} Pytorch  \textbullet{} Tensorflow \\
\sectionsep

\subsection{語学}
日本語 : Advanced (JLPT N1)\\
英 語 : Intermediate (TOEIC L/R 805点)\\
韓国語 : Native
\sectionsep
%%%%%%%%%%%%%%%%%%%%%%%%%%%%%%%%%%%%%%
%     SKILLS
%%%%%%%%%%%%%%%%%%%%%%%%%%%%%%%%%%%%%%

\section{対外活動}

\runsubsection{ヨーロッパ自転車旅}
\descript{| チャレンジ }
\location{2014年4月 - 2014年6月 | 欧州}
\vspace{\topsep}
\begin{tightemize}
\item \textbf{\href{https://nhandsome.github.io/daily/2020/09/17/daily-bicycle-INTRO/}{自転車とテントだけで挑戦した独り旅}}
\end{tightemize}
\sectionsep


\runsubsection{Handium}
\descript{| インターンシップ }
\location{2013年10月 - 2014年3月 | 韓国}
\begin{tightemize}
\item \textbf{\href{https://handium.co.kr/}{カフェ関連スタートアップ}}の商品企画チーム
\item レディ・トゥ・ドリンクやコーヒー新商品企画
\end{tightemize}
\sectionsep


\runsubsection{Global Citizens Youth Assembly}
\descript{| 国際フォーラム }
\location{2012年4月 - 2012年10月 | 韓国・インド}
\begin{tightemize}
\item 持続可能な開発・教育関連の国際フォーラム
\item 青少年が考える持続可能な経済・環境・社会を
話し合い会議、経済チームメンバーとして参加
\end{tightemize}
\sectionsep


\runsubsection{韓国軍隊}
\descript{| 兵役 }
\location{2009年10月 - 2011年7月 | 韓国}
\begin{tightemize}
\item コンピュータ・通信兵
\item 軍隊内のウェブサイト・ネットワーク運用
\end{tightemize}



%%%%%%%%%%%%%%%%%%%%%%%%%%%%%%%%%%%%%%
%
%     COLUMN TWO
%
%%%%%%%%%%%%%%%%%%%%%%%%%%%%%%%%%%%%%%

\end{minipage} 
\hfill
\begin{minipage}[t]{0.55\textwidth} 

%%%%%%%%%%%%%%%%%%%%%%%%%%%%%%%%%%%%%%
%     RESEARCH
%%%%%%%%%%%%%%%%%%%%%%%%%%%%%%%%%%%%%%

\section{研究活動}
\runsubsection{自然言語処理 (NLP)}\\
\location{2020年10月 – 現在 | 九州大学}
\vspace{\topsep} % Hacky fix for awkward extra vertical space
\begin{tightemize}
\item アジア言語のようなLow-Resource Languageに関する
Cross-lingual Word Embedding(CWE)の性能向上を研究
\item 多言語CWEモデルを用いCWEの性能改善させ、単語レベル
翻訳やNLPのDownStream性能を向上することが目標
\item Pytorchベースの教師あり・無し(敵対的学習)学習
\end{tightemize}
\sectionsep

\runsubsection{データサイエンス}
\descript{| 学校・地域企業合同プロジェクト}
\location{2020年10月 – 2021年2月 | 九州大学}
\begin{tightemize}
\item 九州大学主催の\textbf{\href{https://ads.i.kyushu-u.ac.jp/archives/1642}{データ解析PBL成果報告シンポジウム}}
\item 企業の現場ニーズ分析やデータ解析作業に貢献
\item 幼児の足裏スキャン画像からのアーチ形成度の自動判定を
テーマとし、CNN学習・統計的可視化を実現
\end{tightemize}
\sectionsep


%%%%%%%%%%%%%%%%%%%%%%%%%%%%%%%%%%%%%%
%     TECHNICAL EXPERIENCE
%%%%%%%%%%%%%%%%%%%%%%%%%%%%%%%%%%%%%%

\section{職歴}
\runsubsection{Samsung SDS}
\descript{| Software Engineering }
\location{2015年3月 - 2018年3月 | 韓国}
\sectionsep

\descript{FIREWALL POLICY MANAGEMENT SYSTEM PROJECT}
\location{2016年10月 - 2018年3月}
\begin{tightemize}
\item 全社Firewall運用を統合・半自動化させる管理システム開発
プロジェクト
\item アップデート機能のテスト・適用作業及びシステム運用担当
\item Java, Javascript, Jira, Git
\end{tightemize}
\sectionsep

\descript{AMWAY KOREA WEBSITE INTEGRATION PROJECT}
\location{2016年3月 - 2016年10月}
\begin{tightemize}
\item レガシーウェブサイトとオンラインモールの総合・モバイル
システム開発プロジェクト
\item テスト自動化やIssue管理を通じた品質管理サポート担当
\item Selenium, Javascript
\end{tightemize}
\sectionsep

\descript{QUALITY ASSURANCE}
\location{2015年3月 - 2016年3月}
\begin{tightemize}
\item Issueプロジェクトを集中管理・サポートする全社組織
\item Unit・Functionalテストサポート及び役員向けの報告資料
作成担当
\item Selenium, MS PowerPoint
\end{tightemize}
\sectionsep

\sectionsep
\sectionsep
\centering
- 以上- 

\end{minipage} 
\end{document}  \documentclass[]{article}